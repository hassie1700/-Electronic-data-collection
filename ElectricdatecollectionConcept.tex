\documentclass{article} 
\usepackage{graphicx}
\usepackage{float}
\begin{document}
\begin{titlepage}
	\begin{figure}
		\centering
		\includegraphics[height=1.5in]{muk.jpg}
	\end{figure}
	\begin{center}
		\line(1,0){320}\\
		[0.25in]
		\huge{\bfseries ELECTRONIC DATA COLLECTION SYSTEM TO CHECK FOR EXPIRED DRUGS}\\
		[2mm]
		\line(1,0){150}\\
		[1cm]
		\huge{\bfseries CONCEPT PAPER}\\
		[2mm]
		\textsc{\large COLLEGE OF COMPUTING AND INFORMATICS TECHNOLOGY}\\
		[0.5cm]
		\textsc{\large RESEARCH METHODOLOGY}\\
		[0.5cm]
		\textsc{\large By}\\
		[0.5cm]
		\huge{\bfseries FAHAD HASSAN}\\
		[2mm]
		\textsc{\large 13/U/5186/EVE}\\
		[0.5cm]
	\end{center}
	
\end{titlepage}
\thispagestyle{empty}
\section{INTRODUCTION}

The expiry of medicines in Uganda highlights a problem with the supply chain which includes importation, exportation, transportation, medicine selection, quantification, procurement, storage, distribution and use of drugs. An expired drug is the medicine that has come to its end i.e. it should not be used because it may be spoiled, damaged or ineffective due to over staying or even poisonous if any patient takes it.
\section{PROBLEM STATEMENT}
Expired drugs are not only dangerous but also harmful to our bodies. Once taken can cause more side effects that can be or not detected. This as well can lead to the cause of severe pain, prolonged recovery and also many other effects. Considering the increasing levels of self-medication among the elites in Uganda and the lack of implementation measures for self-medication policies, there is no trusted intervention to assist individuals to detect expired drugs.
\section{OBJECTIVES}
\subsection{Main Objective}
To develop an electronic data collection system to detect poor drug description and expired drugs. 

\subsection{Specific Objectives}
To study the existing system which the government uses to trace expired drugs in Uganda in order find the requirements for the new system. 

To implement the proposed expired drug detecting system. 

To test and validate the system

\section{METHODOLOGY}
I will get data from any hospitals or medical center using the phone which uses Odk collect and then upload it to the server for analysis. 
 
\section{SCOPE}
\subsection{Geographical Scope}
The research will be carried out around Kampala district and data will be collected from private medical stores, pharmacies and the national medical store using observations and interviews.

\subsection{Functional Scope}
The system will be an android and web based application that will be able to capture the bar codes on the drugs and detects the expiry dates retrieved from the database.

\subsection{Durational Scope}
The research will run from  April 2017 to May 2017.

\section{SIGNIFICANCE}
The system will be able to detect expired drugs using the barcodes hence giving a feedback to the user whether the drug is expired or not in order not to consume it. 
The system shall be a source of information to healthy officers and other researchers who would like to use the information for other research topics. 
The system shall be of use by all healthy agencies both private and government companies that deal in inspection of expired drugs and personal individuals too. 
The system shall help the government to reduce the circulation of the expired drugs around Kampala since the expiry drug detector will be used in drug stores where drugs are kept before being distributed to various area.

\section{RECCOMENDATION AND CONCLUSION}
I reccomend the implementation of the system to save lives through its use and to take necessary action to doctors how give out such medication.

\section{References}

National Drug Policy and Authority Act 1993.Chapter 206. Retrieved date October 20, 2015, from http://www.ulii.org/ug/legislation/consolidated-act/206.
Nd (2005). The world medicines situation Geneva. Retrieved on October 20 2015,from www.fda.gov/Drugs/.../ucm252375.html.


\end{document}



