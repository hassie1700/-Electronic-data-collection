\documentclass{article} 
\usepackage{graphicx}
\usepackage{float}
\begin{document}
\begin{titlepage}
	\begin{figure}
		\centering
		\includegraphics[height=1.5in]{muk.jpg}
	\end{figure}
	\begin{center}
		\line(1,0){320}\\
		[0.25in]
		\huge{\bfseries ELECTRONIC DATA COLLECTION SYSTEM TO CHECK FOR EXPIRED DRUGS}\\
		[2mm]
		\line(1,0){150}\\
		[1cm]
		\huge{\bfseries FINAL REPORT}\\
		[2mm]
		\textsc{\large COLLEGE OF COMPUTING AND INFORMATICS TECHNOLOGY}\\
		[0.5cm]
		\textsc{\large RESEARCH METHODOLOGY}\\
		[0.5cm]
		\textsc{\large By}\\
		[0.5cm]
		\huge{\bfseries FAHAD HASSAN}\\
		[2mm]
		\textsc{\large 13/U/5186/EVE}\\
		[0.5cm]
	\end{center}
	
\end{titlepage}
\thispagestyle{empty}
\section{CHAPTER ONE}
\subsection{INTRODUCTION}

The expiry of medicines in Uganda highlights a problem with the supply chain which includes importation, exportation, transportation, medicine selection, quantification, procurement, storage, distribution and use of drugs. An expired drug is the medicine that has come to its end i.e. it should not be used because it may be spoiled, damaged or ineffective due to over staying.

According to the National Drug policy and Authority Act (1993) mandates establishment of a national drug authority to ensure the availability of essential, efficacious and cost-effective drugs to the entire population of Uganda, at all times, as a means of providing satisfactory health care and safeguarding the appropriate use of drugs. And it also states that no person or body shall import and export any drugs into Uganda without having a license in relation to their import or export from the National Drug Authority. The license shall be valid for one year and shall state the range of preparations to be imported during that period.

Research done by Katabaazi, Kitutu and Oria (2009) states that the Ugandan pharmaceutical supply system comprises of three non-profit wholesalers (one government medical store and two private non-profit ventures) and several private for-profit wholesale pharmacies that supply medicines in bulk to retail units i.e. private retail pharmacies, hospital pharmacies and drug shops. Drug shops are the smallest retail medicine outlets in the country which are supervised by non-pharmacist health-care professionals who are limited to handling small amounts of over-the-counter medicines.(pg. 154)

Due to lack of discrete bodies to combat loopholes and missing gaps in the supply chain in our country, that's why we have heard many issues of expired drugs being distributed and sold to the public. Sound coordination is needed between public medicine wholesalers and their clients to harmonize procurement and consumption as well as with vertical programs to prevent duplicate procurement.

\subsection{PROBLEM STATEMENT}
Expired drugs are not only dangerous but also harmful to our bodies. Once taken can cause more side effects that can be or not detected. This as well can lead to the cause of severe pain, prolonged recovery and also many other effects. Considering the increasing levels of self-medication among the elites in Uganda and the lack of implementation measures for self-medication policies, there is no trusted intervention to assist individuals to detect expired drugs.
\subsection{OBJECTIVES}
\subsubsection{Main Objective}
To develop an electronic data collection system to gather information on poor drug description and description of expired drugs. 

\subsubsection{Specific Objectives}
To study the existing system which the government uses to collect data on drugs in Uganda in order find the requirements for the new system. 

To implement the proposed drug collection system. 

To test and validate the system.
 
\subsection{SCOPE}
\subsubsection{Geographical Scope}
The data will be collected from private medical stores, pharmacies and the national medical store using the ODK collect software on my phone.

\subsubsection{Functional Scope}
The system will be an android and web based application that will be able to capture the name of the drugs, descriptions, manufacture date and bar codes on the drugs.

\subsubsection{Duration Scope}
The research will run from  April 2017 to May 2017.

\subsection{SIGNIFICANCE}
The system shall be a source of information to healthy officers and other researchers who would like to use the information for other research topics. 
The system shall be of use by all healthy agencies both private and government companies that deal in inspection of expired drugs and personal individuals too. 
The system shall help the government to govern the circulation of the drugs around Kampala.

\section{Chapter two}

\subsection{Literature review}

\subsubsection{Introduction}

This section consists of a critical review of research work from journals, internet sources and other projects already done which are related to the subject area and provides explanations, summaries and critical evaluations on related works done. It contains an analysis of existing literature with the objective of revealing contributions, weaknesses and existing gaps with in some of the current methods used to detect expired drugs.

\subsubsection{Drugs}

�A drug is any chemical you take that affects the way your body works. Alcohol, caffeine, aspirin and nicotine are all drugs.� According to (Who I am,n.d.). A drug must be able to pass from your body into your brain. Once inside your brain, drugs can change the messages your brain cells are sending to each other, and to the rest of your body. They do this by interfering with your brain's own chemical signals: neurotransmitters that transfer signals across synapses.
Some drugs come from plants and others are made in laboratories. The process of making drugs and what goes into them varies from place to place. This makes it very difficult to be sure of what you are actually getting. Street drugs are often mixed with impurities to make them go further, for example: cannabis mixed with other herbs, cocaine and amphetamines mixed with baking powder, anaesthetics, caffeine and heroin.(Where they are and where they come from, n.d.)
particular features of drugs such as molecular weight and size, stereo-isomer, structure-activity relationship, ionization constant, partition coefficient, dosage form, salt preparation, affinity to bind with plasma proteins, receptors, and potency.
Different drugs have different effects, some drugs (such as alcohol, heroin and tranquillisers) have a sedative effect which slow down the way the body and brain function. They can have a numbing effect that produces drowsiness if a lot is taken. Other drugs (such as amphetamine, cocaine, crack and ecstasy) have a stimulant effect giving a rush of energy and making people more alert. A third group of drugs (such as LSD and magic mushrooms and to a lesser extent cannabis and ecstasy) have a hallucinogenic effect. This means they tend to alter the way the user feels, sees, hears, tastes or smells.

\subsubsection*{Drug Monitoring}

These are bodies / organizations that regulate and ensure the safety and effective management of drugs within their own organizations and by anybody or person providing services ensure monitoring and auditing of the management and use of drugs, maintain a record of concerns, regarding relevant individuals, assess and investigate concerns.These institutions are both private and public and they consider the expiry date indicated on the drugs as recommended from the national drugs store. In Uganda the following bodies are responsible for monitoring the quality of drugs in the country; 
Infectious disease institute.
Makerere University College of Healthy Sciences.
Under these organizations, there is a unit called Therapeutic drug monitoring (TDM). Its a branch of clinical chemistry and clinical pharmacology that specializes in the measurement of medication concentrations in blood. Its main focus is on drugs with a narrow therapeutic range, i.e. drugs that can easily be under- or overdosed

\subsubsection{cases studies}

\subsubsection*{Mobile application}
This tracks all drugs within the borders of Turkey with the aim of protecting the community health.  In order to inquiry any drugs using ?TS Mobile application, datamatrix on the drug must be read by the camera using �scan with camera� feature, or barcode and serial numbers must be entered using the �manual inquiry� feature. 
There must be an internet connection for ?TS Mobile application to show the drug�s status. Patients can inquiry drugs using ?TS Mobile application, whenever they want, and get informed, whether those drugs are registered in the Ministry of Health or not. (iTS Mobile, n.d)
\subsubsection*{Technology and the Social Web}
Feeds from pharmaceutical agencies websites and other relevant regulatory agencies are aggregated (fetched, sorted and filtered) in a single view on desktop PC, through browser interface, via mobile devices or using web. 
The technology makes it easy for the public, pharmaceutical companies and law enforcement agencies to keep to date with current trends globally from a single view. Social media tools, such as Facebook, twitter, Google+ etc. are also used for information sharing, emergency warnings and alerts. (iSchool RSS, n.d.)

\subsubsection*{Advantages of existing systems}
Registers the details of the drugs including the expiry date.
Better and improved healthy standards of people due to reduced cases of expired drugs since there governed and their supply is controlled.
Increased quality control of drugs by pharmacies due to focused and strong disciplinary actions.
\subsubsection*{Weaknesses of the existing systems}
The current systems are characterized by manual entry of data and retrieving is done through long procedures, duplication of data since data gets repeated due to inconsistency. Data may also be misplaced or mismatched during entry causing a lot of errors. The systems on the other hand involve a lot of paper work and recordings making it time and space consuming.
\subsubsection{Conclusion}
In conclusion therefore, a lot of literature has been reviewed by many researchers about the concept of drugs however no literature has been reviewed on the control of its supply chain.

After seeing all these difficulties, there is a need for monitoring drug supply for accountability and know when a certain drug is required and in which amounts.

\section{CHAPTER THREE}
\subsection{Methodology}
\subsubsection{Introduction}

This chapter contains the methods that will be used to collect data, methods used in designing the system and how the system that keeps records of all already done projects is implemented. Open Data Kit (ODK) will be used in building the system. It also uses TSDLC which is the structured step-by-step approach of developing systems and consists of Planning, Analysis, Design, Testing, Implementation and Support.

\subsubsection{Data collection methods}

\subsubsection*{Open Data Kit (ODK)}

This is a very flexible system that requires 0.001 percent programming. It has three components - ODK-Build, ODK-Collect and ODK-Aggregate. 
The Build part is an online system used to create an xml form that you can either transfer to your phone or upload to the Aggregate server and have it transfered to your phone.
ODK collect is an app you download from the Google Play Store that you use to collect your data. 
ODK Aggregate provides a ready-to-deploy server and data repository to provide blank forms to ODK Collect (or other OpenRosa clients), accept finalized forms (submissions) from ODK Collect and manage collected data, visualize the collected data using maps and simple graphs, export data (e.g., as CSV files for spreadsheets, or as KML files for Google Earth), and publish data to external systems (e.g., Google Spreadsheets or Google Fusion Tables).
ODK Aggregate can be deployed on Google's App Engine, enabling users to quickly get running without facing the complexities of setting up their own scalable web service. ODK Aggregate can also be deployed locally on a Tomcat server (or any servlet 2.5-compatible (or higher) web container) backed with a MySQL or PostgreSQL database server.


\subsubsection*{Google AppEngine Platform}
To have the system working we will have to build the Aggregate server. This is the most technically challenging part. It involves a number of steps. We use the Google AppEngine platform to build the Aggregate. Google AppEngine provides you with a unique url that can be referenced globally.

\subsubsection{ Data analysis}

The data gathered will be analyzed using Microsoft excel.

\subsubsection{System analysis}

The analysis of user, system, functional and nonfunctional requirements will be conducted in order to review and make sound decisions upon the type of system that will meet the requirements identified.

The systems services, constraints, functionality and goals will be determined and established. The system will come up with satisfactory results basing on the user requirements, the system will mainly focus on drug in the medical centers and the clinical officers. It provides an interface which will enable them interact with the system, availing them with different privileges to enter the data.


\subsubsection{System design}

The design stage includes coming up with an appropriate plan of the architecture that will be used in the development of the system that is based on the documented requirements from the analysis phase. This will be done in two stages which include:

\subsubsection*{Logical design}

Under logical design, a blue print will be developed basing on the documented requirements generated from the analysis phase. This will reflect the desired solution to the problem addressed and also serve as a guide to the actual design of the intended system. Any necessary alterations or a change that will have an impact to the final design will be done in this phase.

The logical design will be developed using data flow diagrams which gives a clear description of the processes involved. Also entity relationship diagrams (E-R) diagrams an enhanced entity relationship diagrams (EER) are to be used, these are diagrams which are based on a perception of the world as consisting of a collection of basic objects (entities) and the relationship among these objects.



\subsubsection*{Physical design}

The objective of this stage is to specify the physical data and process design, using the language and features of the chosen physical environment and incorporating installation standards.

\subsubsection{System implementation}

In implementation, this is where the system is built or designed practically. The system consists of the client server technology and the tools used are;

\subsubsection*{HTML}

This is a web developing and designing language that is used to especially create forms and tables. This language is easy to use, flexible and can be used in corporation with PHP languages. PHP can be embedded within html.

\subsubsection*{Interface development}

On the client side, the android mobile application will be developed using the ODK collect downloaded from the app store. This will provide the interface through which the user will interact with the database to enter the data.

\subsubsection{System validation and testing}

These include the various ways that will prove the efficiency, safety, and effectiveness of the system and they have been categorized into two.

\subsubsection*{Validation}

The system will be designed in such way that there is a small number or no errors at all. For this to happen, checks will always be made so as to have and maintain an accurate output, this system will enable the user to achieve the above by providing user friendly interfaces for in putting data. The output and input interfaces should show error messages whenever errors are encountered. These include wrong entry of data, data that doesn't specify the system requirements. It is through good quality of input that accurate output is obtained hence the validations stated are necessary to ensure accuracy and completeness of data before being entered and stored into the system.


\subsubsection*{Testing}

The system is supposed to be tested before being put in use. The system will be installed and tested on the existing computing infrastructure when testing the system, it should allow the user to add details of the drugs in the database. It will also enable the client to take a snap of the bar cords of the drug.                            
These are some of the screen shots we are having.

 
\section{CHAPTER FOUR}
\subsection{RECOMMENDATION AND CONCLUSION}
I recommend the implementation of the system to provide accountability of the use and provision of the drugs at a certain period of time.

\subsection{References}

National Drug Policy and Authority Act 1993.Chapter 206. Retrieved date October 20, 2015, from http://www.ulii.org/ug/legislation/consolidated-act/206.
Nd (2005). The world medicines situation Geneva. Retrieved on October 20 2015,from www.fda.gov/Drugs/.../ucm252375.html.


\end{document}



